\subsection{Other politeness phenomena}
\label{sec:other}

So far, we have tried to capture one specific area of politeness phenomena: what people infer about white lies given a lay speaker's utterance. However, our model has unlimited possibilities for extending its predictions for other kinds of polite speech, as well as for factors that contribute to the speaker's decision to optimize informativity versus face-saving. In the next subsections, I propose one topic for each category of exciting areas that we would like to examine. 

\subsubsection{Indirect remarks}

There are many different ways in which people speak politely, and one prominent kind besides white lies is indirect remarks. Through indirect remarks, speakers try to convey a particular message in a more nuanced way. This is different from white lies, in that speaker's intentions are no longer hidden, but revealed with suboptimal efficiency. 

\paragraph{Background and model implications} Why would people speak indirectly? Theoretical accounts of indirect speech in situations of potential conflicts argue that indirect language maintains plausible deniability \citep{pinker2008}, and suggests higher stakes for the listener than speaker in case the speaker's wants are not fulfilled \citep{franke2016}. For example, a mobster trying to coerce a restaurant owner into paying protection money, who utters, ``Your daughter is very sweet. She goes to the school in Willow Road, I believe.'' avoids the risk of being sued for threat but also suggests to the owner that his stakes for not paying money are high (his daughter would be in danger). 

Indirect speech in politeness situations, however, is distinguished from speech aimed upon avoiding legal liability or persuading the listener to defer to the speaker's propositions. Why would it be better to say ``I would love another glass of wine, thanks.'' than ``Pour me more wine''? The latter more clearly conveys the guest's intention for the waiter to pour more wine. But the former is less imposing on the waiter, and circumvents an impression that the speaker is in a position to give orders to the listener. Previous empirical work indeed shows that, even when the implied meaning of the requests is the same, people prefer requests whose literal meanings ask for the listener's permission (``Could I ask you where Jordan Hall is?'') to those with literal meanings that assume listener's obligation to respond (``Shouldn't you tell me where Jordan Hall is?'') \citep{clark1980}. 

Our model assumes that a cooperative speaker considers two goals: informativity and face-saving. 
Whereas for white lies, face-saving primarily concerned boosting the listener's self-image, 
for indirect speech, face-saving concerns more the need to avoid impositions on the listener (indirect requests) and to save the speaker's reputation \citep{franke2016} as being genuine and yet well-intentioned. 
For example, hesitantly commenting on an absent colleague's past presentation, ``His presentation wasn't \emph{amazing}...'' may successfully convey the truth but also save the speaker's reputation who cannot be held accountable for explicitly saying ``it was terrible!'' 

There have been some relevant cross-cultural evidence that people do take into account face-informativity tradeoff for polite indirect speech: 
Hebrew adult speakers rate conventional indirect requests as more polite than direct orders or hints, and reason that both face concern and informativity are important for speech to be considered polite \citep{blumkulka1987}; 
similarly, English and Korean adult speakers find evasive remarks to be better than direct or irrelevant remarks \citep{holtgraves1990}.
Finally, \citet{holtgraves2016} suggested that people think of subtle utterances as reflecting varying degrees of both politeness or uncertainty (related to informativity). 

Thus we hypothesize that, similar to white lies, indirect speech reflects speaker's desires to balance between the goal to be informative (convey information in the most direct manner possible) and the goal to save face, this time concerning the speaker's own face as well (maintain her reputation for conveying accurate information with intentions to be polite).

\paragraph{Empirical test: Experiment 8} We will use tasks identical in design to Experiments 3-6, but looking at indirect speech instead of white lies. We will test participants' judgments in contexts in which Bob asks Ann for her opinion on Cayce's cookie. We hypothesize that participants will attribute more niceness but less informativity to Ann when she says ``It wasn't amazing,'' compared to ``It was terrible.''

\subsubsection{Social hierarchy}

So far in our model and empirical work, we have considered what people infer about a lay speaker who is trying to be honest or polite, without providing reasons for why she might want to be polite. There are different sociological variables that determine speaker's intended degree of politeness, and one example is social hierarchy. 

\paragraph{Background and model implications} \citet{Brown1987} proposed that power distance is one of the sociological variables that determines the level of politeness that a speaker would want to use. Thus, the lower the speaker's relative status is with respect to the listener's, the more effort the speaker will exert to be polite by making their utterances more indirect. Indeed, more direct and imposing remarks are associated with higher status of the speaker, and vice versa \citep{holtgraves1986, holtgraves1990, leichty1991}. Even children have  been shown to differentiate between mother and child roles based on utterance politeness \citep{axia1985}. Also, adult speakers have been shown to produce more criticisms when they assume positions of higher status in imaginary scenarios \citep{lim1991}. 

In our proposed work, while supporting \citet{Brown1987}'s initial claim, we would like to elaborate on the workings of the status effect on politeness: 
higher status of the speaker leads to greater emphasis on information exchange than face-saving, 
whereas lower status leads to face-saving prioritized over information exchange. 
Once the knowledge of the interacts' statuses is in common ground, 
the listener will make inferences for utterance meaning and goal parameters based on speaker's status.

\paragraph{Empirical tests: Experiment 9} We will present scenarios that are similar to ones used for Experiments 3-6, but instead of asking for goal and state inferences, we will ask participants to predict the status differences between the speaker and the listener (e.g. ``If one of them is the boss and the other is the employee, who do you think is the boss?''). We hypothesize that a speaker who produces more polite utterances (white lies, indirect remarks) would be inferred to be of lower relative status, and a speaker who produces more direct utterances to be of higher status. 

%%% Local Variables: 
%%% mode: latex
%%% TeX-master: "desc"
%%% End


\subsection{Other politeness phenomena}
\label{sec:other}

So far, we have tried to capture one specific area of politeness phenomena: what people infer about white lies given a lay speaker's utterance. However, our model has unlimited possibilities for extending its predictions for other kinds of polite speech, as well as for factors that contribute to the speaker?s decision to optimize informativity versus face-saving. In the next subsections, I propose one topic for each category of exciting areas to be examined. 

\subsubsection{Indirect remarks}

There are many different ways in which people speak politely, and one prominent kind besides white lies is indirect remarks. Through indirect remarks, speakers try to convey a particular message in a more nuanced way. This is different from white lies, in that speaker?s intentions are no longer hidden, but revealed with suboptimal efficiency. 

\paragraph{Background} Previous empirical work on indirect remarks broadly suggested that people speak indirectly to be polite \citep{clark1980}, and there have been relevant cross-cultural evidence for our argument for face-informativity tradeoff. For example, \citet{blumkulka1987} showed that for Hebrew adult speakers, conventional indirect requests are rated as more polite than direct orders or hints, and reasoned that both face concern and informativity are important for speech to be considered polite. Similarly, \citet{holtgraves1990} found that evasive remarks are better than direct or irrelevant remarks for English and Korean adult speakers. Finally, \citet{holtgraves2016} suggested that people think of subtle utterances as reflecting varying degrees of both politeness or uncertainty (related to informativity). 

\paragraph{Empirical tests} We will use tasks identical in design to Experiments 3-6, but looking at indirect speech instead of white lies. We will test participants' judgments in contexts in which Bob asks Ann for her opinion on Cayce's talk. We hypothesize that participants will attribute more niceness but less informativity to Ann when she says ``It wasn't amazing,'' compared to ``It was terrible.''

\subsubsection{Social hierarchy}

So far in our model and empirical work, we have considered what people infer about a lay speaker who is trying to be honest or polite, without providing reasons for why she might want to be polite. There are different sociological variables that determine speaker's intended degree of politeness, and one example is social hierarchy. 

\paragraph{Background} Brown and Levinson proposed that power distance is one of the sociological variables that determines the level of politeness that a speaker would want to use. Thus, the lower the speaker's relative status is with respect to the listener's, the more effort the speaker will exert to be polite by making their utterances more indirect. Surprisingly little work has been done to confirm this prediction; one study found that direct remarks are associated with higher status of the speaker \citep{holtgraves1986}. 

\paragraph{Empirical tests} We will present scenarios that are similar to ones used for Experiments 3-6, but instead of asking for goal and state inferences, we will ask participants to predict the status differences between the speaker and the listener. We hypothesize that a speaker who produces more polite utterances would be inferred to be of lower relative status, and a speaker who produces more direct utterances to be of higher status. 


%%% Local Variables: 
%%% mode: latex
%%% TeX-master: "desc"
%%% End


\section{Introduction}\label{sec:intro}

Language users hear and produce {\bf polite speech} on a daily basis. But polite utterances seem to often conflict with one important goal of cooperative communication: exchanging information efficiently and accurately \citep{Grice1975}. For example, people produce indirect requests rather than give orders (``It would be great if you could close that window'' as opposed to ``Close that window.''), or white lies to make others feel good (``Your new dress is gorgeous!'')  If information transfer was the only currency in communication, a cooperative speaker would find polite utterances undesirable because they are potentially misleading. 

In our previous work, based on a theory that a cooperative speaker has both an epistemic goal to improve the listener's knowledge state as well as a social goal to minimize any potential damage to the interactants' self-image (face; \citealt{Brown1987}), we argued that language users think about polite speech as reflecting a tradeoff between information transfer and face-saving. We developed a novel computational model (described in sec 3.1) that captures the idea that cooperative speakers attempt to balance between the two goals: information transfer and face-saving. The model was preliminarily supported by US adults' inferential patterns (sec 3.2).

In the current proposal, we propose to perform empirical research that tests our previous modeling work and provides data relevant to expanding it to a much wider range of phenomena. In particular, we have three aims:

\begin{enumerate}

\item Examine {\bf children}'s inferences about polite speech and track the development of polite speech understanding, with the hypothesis that their inferential patterns can be explained using our formal model;

\item Capture {\bf cultural variations} in reasoning about polite speech by testing adult and child participants in India and South Korea;

\item Account for {\bf other politeness phenomena}, such as indirect speech and sociological variables that determine the degree of politeness.

\end{enumerate}

\noindent In sum, with the help of funding from a DDRIG grant, we will be able to complete fieldwork to extend our prior model and provide in-depth understanding of the developmental and cross-cultural trends of various kinds of polite speech phenomena. These efforts will take a concrete step toward quantitative models of the nuances of polite speech, thereby contributing to a richer understanding of pragmatic language processing and production.

%%% Local Variables: 
%%% mode: latex
%%% TeX-master: "desc"
%%% End:


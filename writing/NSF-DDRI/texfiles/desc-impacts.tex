\section{Broader impacts}

\subsection{Enhancing infrastructure for research and education}
Since we have established, and will be establishing even more, new connections for conducting research with adults and children at the three different sites, the proposed work will provide opportunities for other researchers who will take interest in collecting cross-cultural and developmental data.  Also, we will contribute to improving the transparency of data collection and analysis process by pre-registering the proposed studies, and making all the experimental data available publicly on github repositories. Importantly, we will release the model code to the public, to facilitate other researchers' development of formal models of similar phenomena. 

\subsection{Promoting teaching, training, and learning}
The proposed work will provide useful training for Co-PI Yoon and her supervised interns and research assistants, across all three sites of interest. Co-PI Yoon will give guest lectures in undergraduate and graduate courses or seminars that are relevant (e.g. seminars for Center for the Study of Language and Information at Stanford, or CSLI). After graduation, Co-PI Yoon will pursue an academic career (e.g., postdoctoral position) to continue cross-cultural investigations to examine other factors that may contribute to polite language understanding and production. 

\subsection{Promoting underrepresented groups}
The proposed research will help highlight the importance of including underrepresented ethnic groups in empirical research in Linguistics. Looking at cultural and developmental variations together is much needed at this time in language understanding literature, as there are few studies that examine cross-cultural trends in development of high-level language understanding such as polite language comprehension, especially those that account for children in low-income countries such as India. Also, no study to my knowledge has made predictions for different cultures using a theoretical model framework. Thus, the proposed work will broaden the scope of pragmatics literature and open doors to opportunities for more exploration of cross-cultural variations in pragmatic language processing. Furthermore, Co-PI Yoon will mentor students in programs offered by Stanford (e.g. CSLI, Raising Interest in Science and Engineering) that strive to include under-represented groups in the intern pool.
 
\subsection{Benefits to society}
The proposed work is meaningful attempt to facilitate technical development of tactful language production systems, by considering human language user's social goals. Further, we hypothesized that there may be differences across cultures in what is considered to be ``optimal'' communication: some cultures may value honesty over politeness and speak efficiently, directly and explicitly, whereas other cultures may value politeness over honesty and hence speak more subtly and indirectly. This can lead to misunderstandings in intercultural communication, in that the speaker's intentions in one culture may not be as clearly and favorably represented in another. Thus, our proposed work will help advance understanding of how miscommunication may arise between different cultures, due to differences in what are perceived to be optimal tradeoff point between honesty and politeness. Additionally, Co-PI Yoon collaborates with faculty members and students affiliated with Stanford SPARQ (Social Psychological Answers to Real-world Questions) Center, and this work will help promote future collaborations with real-world applications.

\section{Proposed work}
\label{sec:proposed}

We now take up the challenge of developing the RSA framework to
broaden its scope and address the weaknesses described above.
% \Secref{sec:methods} briefly
% reviews our experimental methods.  
Subsections correspond to the broad goals described in
\secref{sec:intro}. Each has a common format: a review of the
limitations of the current RSA framework and then descriptions of the
theoretical extensions we plan to explore and our proposed empirical
tests of the theory.

% CP: I think we don't need to spend the time/space on a review of
% subsections, since the document is short and people tend to skim suc
% paragraphs. I could be convinced otherwise, though.
%
% In our proposed work, we take up the challenge of developing the RSA
% framework to address the weaknesses described above. In the
% following sections, we describe investigations of the correspondence
% of RSA to language production (\secref{sec:proposed-production}),
% extensions to contexts with repeated interactions
% (\secref{sec:proposed-interaction}), modifications to allow for
% bounded rationality on the part of speakers and comprehenders
% (\secref{sec:proposed-bounds}), explorations of the role of
% contextual salience (\secref{sec:proposed-context}), and expansions
% to complex compositional phenomena
% (\secref{sec:proposed-compositionality}). In each of these sections,
% we review the limitations of the current RSA framework, propose
% theoretical extensions, and describe empirical tests of these
% theoretical extensions.

%%% Local Variables: 
%%% mode: latex
%%% TeX-master: "desc"
%%% End


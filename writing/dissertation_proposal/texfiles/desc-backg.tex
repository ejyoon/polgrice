\section{Backgrounds}\label{sec:backg}

Language users hear and produce polite speech on a daily basis. Polite speech ranges from words of apology (``sorry'') or of gratitude (``thanks'') to compliments (``Your new dress is gorgeous!'') and indirect requests (``It would be great if you could pass that salt''). Simple formulaic markers of politeness (e.g. ``sorry,'' ``thanks,'' ``please'') fit in well with theoretical views that polite speech arises from people's tendencies to follow social norms for what constructs `proper' social conduct (e.g., \citealt{ide1989}).

More complicated polite expressions are difficult to explain under these views, however, as the literal meanings of the utterances do not precisely match the speakers' intentions or knowledge states. For example, saying ``Can you speak a little louder?'', when the actual intended meaning is ``Speak louder!'', does not convey the intended meaning in a maximally efficient manner. Also, a compliment such as ``Your new dress is gorgeous!'' seemingly indicates that the dress is literally and truthfully gorgeous, but it is natural to think about situations where speakers do not feel that way and are trying to hide their intentions (``That dress is really ugly'').

As such, polite utterances seem to often conflict with one important goal of cooperative communication: accurate and efficient information transfer \citep{Grice1975}. If information transfer was the only currency in communication, a cooperative speaker would find polite utterances undesirable because they are potentially misleading. People do speak politely, however. Adults and even young children spontaneously produce requests in polite forms \citep{clark1980, axia1985}, and speakers use politeness strategies even while arguing, preventing unnecessary offense to their interactants \citep{holtgraves1997}. Listeners even attribute ambiguous speech to a polite desire to hide a truth that could hurt another's self-image (e.g., \citealt{bonnefon2009}). In fact, it is difficult to imagine a world in which human speech was used purely as a medium for conveying only the truth. Intuitively, politeness is one prominent characteristic that differentiates human speech from stereotyped robotic communication, which may try to follow rules to say ``please'' or ``thanks'' yet still lack genuine politeness.

Do these facts about politeness imply that people are not cooperative communicators in the Gricean sense? \citet{Brown1987} recast the notion of a \emph{cooperative speaker} as one who has both an epistemic goal to improve the listener's knowledge state as well as a social goal to minimize any potential damage to the hearer's (and the speaker's own) self-image, which they called \emph{face}. In their analysis, if the speaker's intended meaning contains no threat to the speaker or listener's face, then the speaker will choose to convey the meaning in an efficient manner, putting it \emph{on the record}. As the degree of face-threat becomes more severe, however, a speaker will choose to be polite by producing more indirect utterances. Saying ``Can you please speak a little louder?'' rather than ``Speak louder!'' is a more indirect form of request instead of order, which gives the listener a sense of autonomy or freedom from imposition and also bestows better reputation upon the speaker herself. Thus, in \citet{Brown1987}'s claim, the motivation for politeness is that deviation from truthfulness of informativity leads to face-saving. 

One possible proposal based on \citet{Brown1987}'s argument is that language users think about polite speech as reflecting a tradeoff between information transfer and face-saving. When a speaker tries to save face, she hides or risks losing information in her intended message by making her utterance false or indirect to some degree. On the other hand, when a speaker prioritizes truthfulness and informativity, she may risk losing the listener's (or the speaker's own) face. In a previous work, my collaborators and I developed a novel computational model (described below) that captures the idea that cooperative speakers attempt to balance between the two goals: information transfer and face-saving. This model builds on a recent formal framework for modeling pragmatic language understanding, the ``rational speech act''  model \citep{goodman2016}. 

%%% Local Variables: 
%%% mode: latex
%%% TeX-master: "desc"
%%% End:


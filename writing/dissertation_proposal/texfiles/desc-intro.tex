\section{Introduction}\label{sec:intro}

Language users hear and produce {\bf polite speech} on a daily basis. But being polite conflicts with one important goal of cooperative communication: exchanging information efficiently and accurately \citep{Grice1975}. To be polite, people produce indirect requests that are much longer than simple imperatives (``It would be great if you could close that window'' as opposed to ``Close that window.''), and tell white lies to make others feel good (``Your new dress is gorgeous!'') Thus, speakers convey information inefficiently and risk losing accurate information (indirect remarks) or even intentionally convey wrong information (lies). If information transfer was the only currency in communication, a cooperative speaker would find polite utterances undesirable because they are potentially misleading. 

Following a classic theory that a cooperative speaker has both an epistemic goal to improve the listener's knowledge state as well as a social goal to minimize any potential damage to the interactants' self-image (\emph{face}; \citealt{Brown1987}), we conceptualize polite speech as reflecting a tradeoff between information transfer and face-saving (\emph{tradeoff hypothesis}). In our previous work, we developed a novel computational model (described in sec 3.1) that captures this tradeoff, providing some support for this model from the judgments of US adults in simple language comprehension tasks (see sec 3.2).

In the current proposal, we propose to perform empirical research that informs basic questions regarding politeness, and also tests our formal model and explores its predictions across much wider range of phenomena. In particular, we have three aims:

\begin{enumerate}

\item Verify tradeoff hypothesis through two case studies of polite speech: {\bf white lies} and {\bf indirect speech};

\item Examine {\bf children}'s inferences about polite speech and track the development of polite speech understanding, with the hypothesis that children's inferential patterns can be explained using our formal model; and

\item Capture {\bf cultural variations} in reasoning about polite speech by comparing adult and child participants in US versus India.

\end{enumerate}

\noindent In sum, I will attempt to provide an in-depth understanding of a range of polite speech phenomena through a formal model and empirical work probing developmental and cross-cultural differences. These efforts take a concrete step toward quantitative models of the nuances of polite speech, thereby contributing to a richer understanding of pragmatic language processing and production more generally.


%%% Local Variables: 
%%% mode: latex
%%% TeX-master: "desc"
%%% End:

